\usepackage{template/style/UNECE2023}
\usepackage{enumerate}
\usepackage{xcolor}
\usepackage{float}
\floatplacement{table}{b}
\definecolor{unece_color}{RGB}{84, 141, 212}

%% the title of your contribution in capital letters:
\newcommand{\TITLE}{\textbf{ASSESSING THE UTILITY OF SYNTHETIC DATA: A DENSITY RATIO PERSPECTIVE} \\}

%% author:
\newcommand{\AUTHOR}{Thom Benjamin Volker (Utrecht University, the Netherlands; Statistics Netherlands, the Netherlands)\\Peter-Paul de Wolf (Statistics Netherlands, the Netherlands)\\Erik-Jan van Kesteren (Utrecht University, the Netherlands)}

%% your organisation
% \newcommand{\ORGANISATION}{\textsuperscript{1} (Utrecht University, the Netherlands), \textsuperscript{2} (Statistics Netherlands, the Netherlands)}
\newcommand{\EMAIL}{\href{mailto:t.b.volker@uu.nl}{t.b.volker@uu.nl}, \href{mailto:pp.dewolf@cbs.nl}{pp.dewolf@cbs.nl}, \href{mailto:e.vankesteren1@uu.nl}{e.vankesteren1@uu.nl}}

%% abstract
\newcommand{\ABSTRACT}{Synthetic data can be a solution to reduce disclosure risks that arise when disseminating research data to the public. However, for the synthetic data to be useful for general inferential purposes, it is paramount that its distribution is similar to the distribution of the observed data. Often, data disseminators consider multiple synthetic data models and make refinements in an iterative fashion. After each adjustment, it is crucial to evaluate whether the quality of the synthetic data has actually improved. Although many methods exist to provide such an evaluation, their results are often incomplete or even misleading. To improve the evaluation strategy for synthetic data, and thereby the quality of synthetic data itself, we propose to use the density ratio estimation framework. Using techniques from this field, we show how an interpretable utility measure can be obtained from the ratio of the observed and synthetic data densities. We show how the density ratio estimation framework bridges the gap between fit-for-purpose and global utility measures, and discuss how it can also be used to evaluate analysis-specific utility. Using empirical examples, we show that density ratio estimation improves on existing (global) utility measures by providing higher statistical power and offering a fine-grained view of discrepancies between the observed and synthetic data. Moreover, we describe several additional advantages of the approach, such as providing a measure of utility on the level of individual synthetic data points, automatic model selection without requiring user specification, and readily available high-dimensional extensions. We conclude that density ratio estimation provides a promising framework in synthetic data generation workflows and present an R-package with functionality to implement the approach.}


